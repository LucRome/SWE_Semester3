%!TEX root = ../../main.tex

\chapter{Funktionale Anforderungen}
\label{sec:funktionale_Anforderungen}
Unter den funktionalen Anforderungen sind alle Anforderungen zu verstehen, die klar definiert sind. Der Stand ihrer Umsetzung kann eindeutig bestimmt werden. \\
Für das Projekt eCourse sind die funktionalen Anforderungen die vom Projektteam umgesetzt werden im folgenden zusammengefasst.\\
\begin{enumerate}
	\item Benutzer- und Rechtesteuerung mit 3 Benutzergruppen
	\begin{enumerate}
		\item \gls{Verwaltungsangestellte} der Hochschule
		\item \gls{Dozierende}
		\item \gls{Studierende}
	\end{enumerate}
	\item a-priori \gls{Landing-Page}, die für alle Besucher einheitlich ist und keine sensiblen Daten enthält
	\item Login und Logout
	\begin{enumerate}
		\item erfolgreicher Login leitet auf die a-posteriori \gls{Landing-Page} weiter
		\item erfolgloser Login führt zu einer Fehlermeldung
		\item erfolgreicher Logout führt zurück auf die a-priori \gls{Landing-Page}
	\end{enumerate}
	\item a-posteriori \gls{Landing-Page}, die für alle Benutzergruppen individuell ist und an ihre Bedürfnisse angepasst ist, aber für alle Benutzergruppen eine Kursübersicht enthält
	\item Studierende haben von der Kursübersicht zugriff auf eine Klausurübersicht oder auf eine \gls{kursspezifische} Seite
	\item \gls{Studierende} haben sowohl in der Kursübersicht, als auch in der Klausurübersicht und den \gls{kursspezifische}n Seiten die Möglichkeit eine bearbeitete Abgabe zu einer bestimmten Aufgabe hochzuladen
	\item \gls{Studierende} haben sowohl in der Kursübersicht, als auch in der Klausurübersicht und den \gls{kursspezifische}n Seiten die Möglichkeit eine bestimmte Aufgabe herunterzuladen
	\item \gls{Dozierende} können von der Kursübersicht aus einen \gls{Kurs} anlegen
	\item \gls{Dozierende} können von der Kursübersicht aus einen \gls{Kurs} bearbeiten, dies impliziert auch, \gls{Aufgabe}n für diesen \gls{Kurs} hochzuladen
	\item \gls{Dozierende} können von der Kursübersicht auf eine Klausurübersicht gelangen
	\item \gls{Dozierende} können in der Klausurübersicht die \gls{bearbeitete}n Abgaben der \gls{Studierende}n herunterladen
	\item \gls{Verwaltungsangestellte} der Hochschule können ausgehend von der Kursübersicht verschiedene Unterseiten aufrufen
	\begin{enumerate}
		\item \gls{Kurs}e anlegen
	 	\item Gruppen von \gls{Studierende}n anlegen
		\item Kursgruppen anlegen
		\item \gls{Kurs}e bearbeiten (bearbeiten, löschen)
		\item Benutzer verwalten (anlegen, bearbeiten, löschen)
	\end{enumerate}	 
	\item die Datenbank, die an das \gls{Back-End} anschließt soll gegen verschiedene \gls{relationale Datenbank}en austauschbar sein
\end{enumerate}

\textcolor{magenta}{Hier noch weitere funktionale Anforderungen einfügen}