%!TEX root = ../../main.tex

\chapter{Funktionale Anforderungen}
\label{sec:funktionale_Anforderungen}
Unter den funktionalen Anforderungen sind alle Anforderungen zu verstehen, die klar definiert sind. Der Stand ihrer Umsetzung kann eindeutig bestimmt werden. \\
Für das Projekt eCourse sind die funktionalen Anforderungen die vom Projektteam umgesetzt werden im folgenden zusammengefasst.\\
\begin{enumerate}
	\item Benutzer- und Rechtesteuerung mit 3 Benutzergruppen
	\begin{enumerate}
		\item Verwaltungsangestellte der Hochschule
		\item Dozierende
		\item Studierende
	\end{enumerate}
	\item a-priori Landing-Page, die für alle Besucher einheitlich ist und keine sensiblen Daten enthält
	\item Login und Logout
	\begin{enumerate}
		\item erfolgreicher Login leitet auf die a-posteriori Landing-Page weiter
		\item erfolgloser Login führt zu einer Fehlermeldung
		\item erfolgreicher Logout führt zurück auf die a-priori Landing-Page
	\end{enumerate}
	\item a-posteriori Landing-Page, die für alle Benutzergruppen individuell ist und an ihre Bedürfnisse angepasst ist, aber für alle Benutzergruppen eine Kursübersicht enthält
	\item Studierende haben von der Kursübersicht zugriff auf eine Klausurübersicht oder auf eine kursspezifische Seite
	\item Studierende haben sowohl in der Kursübersicht, als auch in der Klausurübersicht und den kursspezifischen Seiten die Möglichkeit eine bearbeitete Abgabe zu einer bestimmten Aufgabe hochzuladen
	\item Studierende haben sowohl in der Kursübersicht, als auch in der Klausurübersicht und den kursspezifischen Seiten die Möglichkeit eine bestimmte Aufgabe herunterzuladen
	\item Dozierende können von der Kursübersicht aus einen Kurs anlegen
	\item Dozierende können von der Kursübersicht aus einen Kurs bearbeiten, dies impliziert auch, Aufgaben für diesen Kurs hochzuladen
	\item Dozierende können von der Kursübersicht auf eine Klausurübersicht gelangen
	\item Dozierende können in der Klausurübersicht die bearbeiteten Abgaben der Studierenden herunterladen
	\item Verwaltungsangestellte von der Hochschule können aussgehend von der Kursübersicht verschiedene Unterseiten aufrufen
	\begin{enumerate}
		\item Kurse anlegen
	 	\item Gruppen von Studierenden anlegen
		\item Kursgruppen anlegen
		\item Kurse bearbeiten (bearbeiten, löschen)
		\item Benutzer verwalten (anlegen, bearbeiten, löschen)
	\end{enumerate}	 
	\item die Datenbank, die an das Backend anschließt soll gegen verschiedene relationale Datenbanken austauschbar sein
\end{enumerate}

\textcolor{magenta}{Hier noch weitere funktionale Anforderungen einfügen}