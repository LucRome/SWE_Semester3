%!TEX root = ../main.tex

%
% To create glossary run the following command: 
% makeglossaries main.acn && makeglossaries main.glo
%

%
% Glossareintraege --> referenz, name, beschreibung
% Aufruf mit \gls{...}
%
\newglossaryentry{Glossareintrag}{name={Glossareintrag},plural={Glossareinträge},description={Ein Glossar beschreibt verschiedenste Dinge in kurzen Worten}}

\newglossaryentry{GitHub}{name={GitHub}, plural = {}, description={GitHub ist ein auf Git-basierender Onlinedienst zur Versionsverwaltung von Software-Projekten.}}

\newglossaryentry{Repository}{name={Repository}, plural={Repositories},description={Ein Repository ist ein verwaltetes Verzeichnis. In diesem Kontext enthält es den Soucecode des Projekes.}}

\newglossaryentry{Front-End}{name={Front-End}, plural={}, description={Beschreibt einen Teil einer Schichteneinteilung. Dabei steht das Front-End näher an Benutzer und Eingabe. Zugehöriger Begriff: Back-End}}

\newglossaryentry{Back-End}{name={Back-End}, plural={}, description={Beschreibt einen Teil einer Schichteneinteilung. Dabei steht das Back-End näher an System und Verarbeitung. Zugehöriger Begriff: Front-End}}

\newglossaryentry{Development Team}{name={Developmenmt Team}, plural={}, description={Das Development Team umfasst alle Projektmitglieder, die sich um die Entwicklung neuer Software kümmern.}}

\newglossaryentry{Daily Scrum}{name={Daily Scrum}, plural={}, description={Das Daily Scrum ist ein kurzes in diesem Kontext Montags und Mittwochs stattfindendes Meeting, in dem jedes Mitglied des Development Teams kurz über seine Arbeit seit dem letzten Daily Scrum und bis zum nächsten Daily Scrum berichtet.}}

\newglossaryentry{Sprint Review}{name={Sprint Review}, plural={}, description={Das vorletzte Meeting eines Sprints ist die Sprint Review. Hier wird über die Ergebnisse des letzten Sprints gesprochen.}}

\newglossaryentry{Sprint Retrospektive}{name={Sprint Retrospektive}, plural={}, description={Das letzte Meeting eines Sprints ist die Sprint Retrospektive. Hier reflektiert das Projektteam über die Arbeitsweise im letzten Sprint.}}

\newglossaryentry{Sprint Planning}{name={Sprint Planning}, plural={}, description={Das erste Meeting eines Sprints ist das Sprint Planning. Hier definiert das Team das Arbeitsvolumen des Sprints.}}

\newglossaryentry{Kanban-Board}{name={Kanban-Board}, plural={}, description={Das Kanban-Board ist eine Methode zur Visualisierung des vorliegenden Arbeitspensums.}}

\newglossaryentry{Issue}{name={Issue}, plural={Issues}, description={Ein Issue beschreibt eine Verbesserung, eine Aufgabe oder eine Frage zum zugehörigen Repository.}}

\newglossaryentry{Pull-Request}{name={Pull-Request}, plural={}, description={Ein Pull-Request ist eine Änderung des Quellcodes, die in Hauptversion des Quellcodes aufgenommen werden soll. Eine solche Änderung muss durch mindestens ein Mitglied des Projektteams abgenommen werden.}}

\newglossaryentry{Scrum}{name={Scrum}, plural={}, description={Scrum ist eine agile Arbeitsweise zur Softwareentwicklung.}}

\newglossaryentry{Sprint}{name={Sprint}, plural={}, description={Ein Sprint repräsentiert einen Iterationsschritt im Scrum-Prozess.}}