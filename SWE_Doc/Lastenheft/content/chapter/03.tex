%!TEX root = ../../main.tex

\chapter{Soll-Konzept}
\section{MUST}
Das Projektergebnis muss folgende Anforderungen umsetzen: \\ \\
\textbf{Drei Benutzergruppen} \\
Es gibt die Benutzergruppen: Dozierende, Studierende und Administratoren \\
Die Gruppe der Dozierenden kann folgende Aktionen ausführen: 
\begin{itemize}
\item Kurse erstellen
\item Studierende zu Kursen hinzufügen 
\item Aufgaben in den Kurs hochladen
\item die Aufgaben mit einer Bearbeitungszeit versehen
\item bearbeitete Abgaben von Studenten herunterladen
\item korrigierte Abgaben hochladen
\end{itemize}
Die Gruppe der Studenten kann folgende Aktionen ausführen:
\begin{itemize}
\item zur Verfügung stehende Aufgaben herunterladen
\item bearbeitete Abgaben hochladen
\item korrigierte Abgaben einsehen
\end{itemize}
Die Gruppe der Administratoren kann folgende Aktionen ausführen:
\begin{itemize}
\item Kurse erstellen
\item Studierende zu Kursen hinzufügen
\item Dozierende zu Kursen hinzufügen
\item Studierende aus Kursen löschen
\item Kurse löschen
\item Studierende und Dozierende aus der Datenbank löschen
\end{itemize}

\textbf{Benutzeraktionen} \\
Jeder Benutzer soll in der Lage sein folgende Aktionen auszuführen:
\begin{itemize}
\item eigenes Password ändern
\end{itemize}

\textbf{Login} \\
Einem Nutzer werden bei Aufruf der Seite keine sensitiven Informationen angezeigt. \\
Es soll eine Seite angezeigt werden, die nur generelle Informationen enthält, die in keiner Verbindung zu den Nutzern der Seite stehen. \\
Erst nach einem Login mit Benutzername und Passwort soll eine Übersichtsseite angezeigt werden, die auf den Nutzer entsprechend seiner Nutzergruppe zugeschnitten ist. \\

\section{SHOULD}
Das Projektergebnis sollte folgende Anforderungen umsetzten:
\begin{itemize}
\item Der Administrator soll Studenten zu Gruppen zusammenfügen können
\item Die Namen der Kurse werden automatisch generiert
\item Dozierende sollen die Bewertung für bearbeitete Abgaben von Studierenden eintragen können
\item Studierende sollen die Bewertung für ihre bearbeitete Abgaben einsehen können
\item Administratoren sollen die eingetragene Bewertung der Abgaben von Studierenden einsehen und freigeben können
\end{itemize}

Anmerkung: Studierende sollen Bewertungen von Aufgaben erst dann einsehen, wenn diese von einem Administrator freigeschaltet wurden

\section{COULD}
Das Projektergebnis kann folgende Anforderungen umsetzen:
\begin{itemize}
\item Ein*e Dozent*in wird nach einer definierten Zeit automatisch an die Bewertung von Abgaben erinnert
\item ein Logo für die Software
\item eine Chatfunktion zwischen Studierenden und Dozierenden
\item Dozierende können neben Aufgaben auch Vorlesungsmaterialien hochladen
\end{itemize}





