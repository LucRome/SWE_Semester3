%!TEX root = ../../main.tex

\chapter{Soll-Konzept}
\section{MUST}
Das Projektergebnis muss folgende Anforderungen umsetzen: \\ \\
\textbf{Drei Benutzergruppen} \\
Es gibt die Benutzergruppen: Dozent, Student und Administrator \\
Die Gruppe Dozent kann folgende Aktionen ausführen: 
\begin{itemize}
\item Kurse erstellen
\item Studenten zu Kursen hinzufügen 
\item Aufgaben in den Kurs hochladen
\item die Aufgaben mit einer Bearbeitungszeit versehen
\item bearbeitete Abgaben von Studenten einsehen und herunterladen
\item korrigierte Abgaben hochladen
\item die Bewertung für Abgaben eintragen
\end{itemize}
Die Gruppe der Studenten kann folgende Aktionen ausführen:
\begin{itemize}
\item zur Verfügung stehende Aufgaben einsehen und herunterladen
\item bearbeitete Aufgaben hochladen
\item korrigierte Abgaben einsehen
\item Bewertung der Abgaben einsehen
\end{itemize}
Die Gruppe der Administratoren kann folgende Aktionen ausführen:
\begin{itemize}
\item Kurse erstellen
\item Studenten zu Kursen hinzufügen
\item Dozenten zu Kursen hinzufügen
\item bearbeitete Abgaben von Studenten einsehen und herunterladen
\item korrigierte Abgaben einsehen und herunterladen
\item die eingetragene Bewertung der Abgaben einsehen und freigeben
\item Studenten aus Kursen löschen
\item Kurse löschen
\item Studenten und Dozenten aus der Datenbank löschen
\end{itemize}

Anmerkung: Studenten sollen Bewertungen von Aufgaben erst dann einsehen, wenn diese von einem Administrator freigeschaltet wurden

\textbf{Benutzeraktionen} \\
Jeder Benutzer soll in der Lage sein folgende Aktionen auszuführen:
\begin{itemize}
\item eigenes Password ändern
\end{itemize}

\textbf{Login} \\
Einem Nutzer werden bei Aufruf der Seite keine sensitiven Informationen angezeigt. \\
Es soll eine Seite angezeigt werden, die nur generelle Informationen enthält, die in keiner Verbindung zu den Nutzern der Seite stehen. \\
Erst nach einem Login mit Benutzername und Passwort soll eine Übersichtsseite angezeigt werden, die auf den Nutzer entsprechend seiner Nutzergruppe zugeschnitten sind. \\

\section{SHOULD}
Das Projektergebnis sollte folgende Anforderungen umsetzten:
\begin{itemize}
\item Der Administrator soll Studenten zu Gruppen zusammenfügen können
\item Die Namen der Kurse werden automatisch generiert
\end{itemize}

\section{COULD}
Das Projektergebnis kann folgende Anforderungen umsetzen:
\begin{itemize}
\item Ein Dozent wird nach einer definierten Zeit automatisch an die Bewertung von Abgaben erinnert
\item ein Logo für die Software
\item eine Chatfunktion zwischen Studenten und Dozenten
\item Dozenten können neben Aufgaben auch Vorlesungsmaterialien hochladen
\end{itemize}





