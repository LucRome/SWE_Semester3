%!TEX root = ../../main.tex

\chapter{Einsatzumgebung und Einsatzzweck}
Die im Projekt entstehende Software soll für den Einsatz in folgenden Browser geeignet sein: Firefox, Chrome und Opera. \\
Die an das \gls{Backend} angeschlossene Datenbank soll variabel sein, da hier noch keine eindeutige Festlegung auf eine konkrete Datenbank stattgefunden hat. Lediglich die Art der Datenbank steht fest. Es muss eine \gls{relationale Datenbank} sein. Die Datenbanksprache ist \gls{SQL}.\\

Die Software wird als Verbindung zwischen \gls{Studierende}n, \gls{Dozierende}n und den \gls{Verwaltungsangestellte}n der Hochschule genutzt. Sie dient der Verwaltung von Studierendenkursen und den dabei anfallenden Dateien und Daten, die zwischen \gls{Studierende}n und \gls{Dozierende}n bzw. zwischen \gls{Studierende}n / \gls{Dozierende}n und der Verwaltung ausgetauscht werden müssen. \\
Darunter zählen beispielsweise Klausuren und die dazugehörigen \gls{bearbeitete} Abgaben. Weitere Angaben dazu sind in Kapitel \ref{sec:funktionale_Anforderungen} zu finden. 
