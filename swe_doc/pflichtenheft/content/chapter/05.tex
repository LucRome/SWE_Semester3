\chapter{Risiken und Risikobewertung}
Bei der Durchführung des Projekts können verschiedene Risiken auftreten. \\
Diese Risiken sind in Tablelle \ref{tab:Risiken} zusammengefasst. \\
Die Bewertung der Risiken durch das Projektteam und die dadurch getroffenen Entscheidung zur Behandlung der Risiken sind im Projekthandbuch Kapitel 6 Risikomanagement zu finden.
\begin{table}
\centering
\scriptsize
\begin{tabularx}{\textwidth}{|l|l|X|} 
\hline
\textbf{Ursache} & \textbf{Ausmaß} & \textbf{Risiko} \\
\hline
Anforderungen & Produkt & Ungenügende Anforderungsanalyse  \\ 
\hline
 &  & Im Produkteinsatz stellt sich heraus, dass mit wachsender Benutzerzahl Einbrüche hinsichtlich der Performance einhergehen. Als Folge können nicht alle Funktionen effektiv und effizient genutzt werden. Die Nutzvorteile sind eingeschränkt.  \\ 
\hline
Extern & Produkt & Höhere Gewalt beeinflusst das Projekt  \\ 
\hline
Kommunikation & Team & Probleme werden nicht offen angesprochen \\ 
\hline
 &  & Probleme bei der Programmierung werden zu lange nicht angesprochen \\ 
\hline
 &  & Fehlende Dokumente\\ 
\hline
Management & Produkt & unrealistische Zeitplanung \\ 
\hline
 &  & fehlende Dokumentation \\ 
\hline
 &  & vergessene und nachträglich hinzugefügte Tasks \\ 
\hline
 &  & zu viel Kommunikation\\ 
\hline
 & Team & schlechter Informationsfluss \\ 
\hline
 &  & Unklare Befugnisse: Es ist unklar wer die Autorität hat ein Projektziel durchzusetzen  \\ 
\hline
Produktumfang & Produkt & Komplexität wird unterschätzt \\ 
\hline
 &  & Im Zeitverlauf über die Entwicklungsphasen wird keine klare Linie verfolgt sondern es gibt ständig neue und wechselnde Wünsche nach Funktionen; die an die Software gestellten Anforderungen werden kontinuierlich verändert\\ 
\hline
 &  & Nicht umsetzbares Design  \\ 
\hline
 &  & Komponenten arbeiten nicht zusammen\\ 
\hline
 &  & Nur teilweise erfahren Software-Entwickler  \\ 
\hline
 &  & Komponenten laufen nicht stabil  \\ 
\hline
 &  & Probleme mit Projektmanagemensystem \\ 
\hline
 & Steakholder & Sicherheitslücken\\ 
\hline
 &  & Testumgebung für Integrationstests sind nicht verfügbar  \\ 
\hline
Team & Produkt & Nur teilweise erfahren Software-Entwickler  \\ 
\hline
 &  & Mangelhafte Zeitplanung   \\ 
\hline
 &  & Implementierung unnötiger Eigenschaften und keine Entwicklung einer der Anforderung entsprechenden Eigenschaft  \\ 
\hline
 &  & Nicht genügend Entwickler  \\ 
\hline
 &  & Austausch/ Wegfall von Personal (Exmatrikulation o. ä.)   \\ 
\hline
 &  & Einsatz neuer, unbekannter Tools  \\ 
\hline
 &  & Mitarbeiter müssen sich in Tools einarbeiten  \\ 
\hline
 &  & Mangelnde Leistung/Engagement  \\ 
\hline
 &  & Negative Einstellung gegenüber des Projektes  \\ 
\hline
 & Team & Geringe Motivation\\ 
\hline
\end{tabularx}
\caption{Risiken im Projekt eCourse}
\label{tab:Risiken}
\end{table}