%!TEX root = ../main.tex

%
% To create glossary run the following command: 
% makeglossaries main.acn && makeglossaries main.glo
%

%
% Glossareintraege --> referenz, name, beschreibung
% Aufruf mit \gls{...}
%

\newglossaryentry{Backend}{name={Backend},plural={},description={Beschreibt einen Teil einer Schichteneinteilung. Dabei steht das Backend näher an System und Verarbeitung. Zugehöriger Begriff: Frontend}}

\newglossaryentry{Frontend}{name={Frontend},plural={},description={Beschreibt einen Teil einer Schichteneinteilung. Dabei steht das Frontend näher an Benutzer und Eingabe. Zugehöriger Begriff: Backend}}

\newglossaryentry{relationale Datenbank}{name={relationale Datenbank},plural={},description={Datenbank, die auf einem tabellenbasierten relationalen Datenbankmodell beruht}}

\newglossaryentry{Studierende}{name={Studierende},plural={},description={Personen, die an einer Hochschule eingeschrieben sind und dort eine akademische Aus- bzw. Weiterbildung erhalten}}

\newglossaryentry{Dozierende}{name={Dozierende},plural={},description={Personen die an einer Hochschule lehren}}

\newglossaryentry{Verwaltungsangestellte}{name={Verwaltungsangestellte der Hochschule},plural={},description={Personen, die an einer Hochschule angestellt sind und dort verwaltende Tätigkeiten ausführen}}

\newglossaryentry{Kurs}{name={Kurs},plural={},description={Bildet die Mitglieder und in Teilen auch die Inhalte eines realen Vorlesungskurses online ab}}

\newglossaryentry{Aufgabe}{name={Aufgabe},plural={},description={Vorlesungsfachspezifische Klausuren und Übungsblätter die von Studierenden bearbeitet werden sollten}}

\newglossaryentry{bearbeitete}{name={Bearbeitete Abgabe},plural={},description={Von Studierenden erstellte Lösungen für eine bestimmte Aufgabe}}

\newglossaryentry{SQL}{name={SQL},plural={},description={Structured Query Language, Datenbanksprache für relationale Datenbanken}}

\newglossaryentry{Landing-Page}{name={Landing-Page},plural={},description={Erste Unterseite, die beim Aufruf der Webseite angezeigt wird }}

\newglossaryentry{kursspezifische}{name={kursspezifische Seite},plural={},description={Enthält alle nutzerspezifischen Informationen über einen bestimmten Kurs bei dem dieser Nutzer angemeldet ist}}