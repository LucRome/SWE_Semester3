%!TEX root = ../main.tex

%
% To create glossary run the following command: 
% makeglossaries main.acn && makeglossaries main.glo
%

%
% Glossareintraege --> referenz, name, beschreibung
% Aufruf mit \gls{...}
%



\newglossaryentry{Aufgabe}{name={Aufgabe},plural={},description={Vorlesungsfachspezifische Klausuren und Übungsblätter die von Studierenden bearbeitet werden sollten}}

\newglossaryentry{bearbeitete Abgabe}{name={Bearbeitete Abgabe},plural={},description={Von Studierenden erstellte Lösungen für eine bestimmte Aufgabe}}


\newglossaryentry{Klausur}{name={Klausur},plural={Klausuren},description={Eine Klausur ist eine schriftliche Prüfungsleistung, die Studierende meist am Ende des Semesters erbringen müssen}}

\newglossaryentry{Studierende}{name={Studierende},plural={},description={Personen, die an einer Hochschule eingeschrieben sind und dort eine akademische Aus- bzw. Weiterbildung erhalten}}

\newglossaryentry{Dozierende}{name={Dozierende},plural={},description={Personen die an einer Hochschule lehren}}

\newglossaryentry{Verwaltungsangestellte der Hochschule}{name={Verwaltungsangestellte der Hochschule},plural={},description={Personen, die an einer Hochschule angestellt sind und dort verwaltende Tätigkeiten ausführen}}

\newglossaryentry{Kurs}{name={Kurs},plural={},description={Bildet die Mitglieder und in Teilen auch die Inhalte eines realen Vorlesungskurses online ab}}


\newglossaryentry{Backend}{name={Backend},plural={},description={Beschreibt einen Teil einer Schichteneinteilung. Dabei steht das Backend näher an System und Verarbeitung. Zugehöriger Begriff: Frontend}}

\newglossaryentry{Frontend}{name={Frontend},plural={},description={Beschreibt einen Teil einer Schichteneinteilung. Dabei steht das Frontend näher an Benutzer und Eingabe. Zugehöriger Begriff: Backend}}

\newglossaryentry{Standalone}{name={Standalone},plural={Standalone},description={Standalone bezeichnet hier Software die keine Schnittstellen zu anderen Anwendungen hat}}

\newglossaryentry{Webinterface}{name={Webinterface},plural={Webinterface},description={Das Webinterface ist die Schnittstelle zum dahinterliegenden System über das Hypertext Transfer Protocol}}

\newglossaryentry{Framework}{name={Framework},plural={Framework},description={Ein Softwaregerüst, dass bei der Programmierung eingesetzt wird und dem Programmierer einen Rahmen zur Verfügung stellt, innerhalb dessen er entwickelt}}