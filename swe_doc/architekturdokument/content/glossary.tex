%!TEX root = ../main.tex

%
% To create glossary run the following command: 
% makeglossaries main.acn && makeglossaries main.glo
%

%
% Glossareintraege --> referenz, name, beschreibung
% Aufruf mit \gls{...}
%
\newglossaryentry{Glossareintrag}{name={Glossareintrag},plural={Glossareinträge},description={Ein Glossar beschreibt verschiedenste Dinge in kurzen Worten}}

\newglossaryentry{Aufgabe}{name={Aufgabe},plural={Aufgaben},description={Ein Glossar beschreibt verschiedenste Dinge in kurzen Worten}}

\newglossaryentry{bearbeitete Abgabe}{name={bearbeitete Abgabe},plural={bearbeitete Abgabe},description={Ein Glossar beschreibt verschiedenste Dinge in kurzen Worten}}

\newglossaryentry{Klausur}{name={Klausur},plural={Klausuren},description={Ein Glossar beschreibt verschiedenste Dinge in kurzen Worten}}

\newglossaryentry{Dozierende}{name={Dozierende},plural={Dozierenden},description={Ein Glossar beschreibt verschiedenste Dinge in kurzen Worten}}

\newglossaryentry{Studierende}{name={Studierende},plural={Studierende},description={Ein Glossar beschreibt verschiedenste Dinge in kurzen Worten}}

\newglossaryentry{Verwaltungsangestellte der Hochschule}{name={Verwaltungsangestellte der Hochschule},plural={Verwaltungsangestellte der Hochschule},description={Ein Glossar beschreibt verschiedenste Dinge in kurzen Worten}}

\newglossaryentry{Kurs}{name={Kurs},plural={Kurse},description={Ein Glossar beschreibt verschiedenste Dinge in kurzen Worten}}

\newglossaryentry{Backend}{name={Backend},plural={Backend},description={Ein Glossar beschreibt verschiedenste Dinge in kurzen Worten}}

\newglossaryentry{Frontend}{name={Frontend},plural={Frontend},description={Ein Glossar beschreibt verschiedenste Dinge in kurzen Worten}}

\newglossaryentry{Standalone}{name={Standalone},plural={Standalone},description={Ein Glossar beschreibt verschiedenste Dinge in kurzen Worten}}

\newglossaryentry{Framework}{name={Framework},plural={Framework},description={Ein Glossar beschreibt verschiedenste Dinge in kurzen Worten}}