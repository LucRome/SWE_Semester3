\chapter{Lösungsstrategien}
Zur Implementierung der Anwendung werden im Folgenden Lösungsstrategien und Entscheidungen aufgeführt. Diese basieren auf der Aufgabenstellung (siehe Kapitel \ref{sec:Aufgabe})sowie den Qualitätszielen (siehe Kapitel \ref{sec:Quali}).

Eine wichtige Grundentscheidung für die Entwicklung einer Software ist die Einigung auf eine bzw. mehrere Programmiersprachen für die unterschiedlichen Bereiche des Systems. Verwendet wurden vor allem die Sprachen HTML und Python. Die weiteren Sprachen, die einen geringeren Teil zur Entwicklung des Projektes beitragen sind dem Kapitel \ref{sec:Entwurf} zu entnehmen. \\
Zur Umsetzung der verschiedenen Anforderungen hat sich das Team dazu entschlossen nach den Prinzipien von Scrum zu arbeiten. Um den Scrum-Prozess effizienter zu gestalten wurde das Entwicklungsteam zusätzlich in Expertengruppen für \gls{Frontend}, \gls{Backend} und Dokumentation unterteilt. Genauere Ausführungen über den Scrum-Prozess und die vom Entwicklungsteam getroffenen Entscheidungen sind im Projekthandbuch zu finden.
Zur Umsetzung der Qualitätsanforderungen dient vor allem eine hohe Testabdeckung, ebenso wie ein hohes Maß an Kommunikation innerhalb des Teams.