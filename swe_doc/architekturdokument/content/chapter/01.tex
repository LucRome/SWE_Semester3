%!TEX root = ../../main.tex

\chapter{Einführung und Ziele}
Die Anwendung eCourse dient dem Austausch von \gls{Aufgabe}n und \gls{Klausur}en sowie deren \gls{bearbeitete Abgabe}n zwischen \gls{Dozierende}n und \gls{Studierende}n. Außerdem können \gls{Verwaltungsangestellte der Hochschule} \gls{Kurs}e und Benutzer verwalten und haben auch Einsicht in die ausgetauschten Dateien. \\
Dadurch soll der Studienalltag an einer Hochschule erleichtert werden. Der Austausch von Dateien muss nicht mehr per E-Mail erfolgen. \\
Besonders in der aktuell anhaltenden Corona-Pandemie, erleichtert dies den Hochschulalltag, da \gls{Klausur}en dann über die Anwendung eCourse verwaltet werden können.  

\section{Aufgabenstellung}
\label{sec:Aufgabe}
Vom Kunden wurde die Aufgabe gestellt, eine webbasierte Anwendung mit Datenbankanbindung und verschiedenen Nutzerrollen zu entwerfen. \\
Darauf aufbauend wurde das Ziel gefasst, eine Anwendung zu entwerfen, über die Dokumente zwischen \gls{Studierende}n und \gls{Dozierende}n einer Hochschule ausgetauscht werden können. Die dafür zu Grunde liegenden Anforderungen sind im Lastenheft zusammengefasst.

\section{Qualitätsziele}
\label{sec:Quali}
Um die Software für alle beteiligten Anwender so gut wie möglich zu gestalten wurden verschiedene Qualitätsziele definiert. Diese sind in \ref{tab:Qualitätsziele} zusammengefasst. Diese Qualitätsziele sind auch für die grundlegenden Architekturentscheidungen von besonderer Wichtigkeit.

\begin{table}[H]
\centering
\begin{tabularx}{\textwidth}[H]{|c|c|X|}
\hline
Priorität &	Ziel & Szenario\\
\hline
1 & Korrektheit & Software folgt Benutzereingabe, es sei denn dieser verletzt die Zugriffsrechte \\
\hline
2 & Konsistenz & Gespeicherte Daten sind eindeutig und nicht redundant \\
\hline
3 & Benutzerfreundlichkeit & Ansprechendes und intuitives User-Interface \\
\hline
4 & Zuverlässigkeit & eCourse läuft sicher und ist frei von Abstürzen \\
\hline
5 & Portabilität & Hardware lässt sich (mit akzeptablem Aufwand) an einen anderen Standort bringen\\
\hline
\end{tabularx}
\caption{Qualitätsziele im Projekt eCourse}
\label{tab:Qualitätsziele}
\end{table}

\section{Stakeholder}

\begin{table}[H]
\centering
\begin{tabularx}{\textwidth}[H]{|c|X|X|}
\hline
Rolle &	Kontakt & Erwartungshaltung\\
\hline
Kunde & Stephan.Brunnet@softwareinmotion.de & funktionsfähige Anwendung und gute Dokumentation des Projekts \\
\hline
Entwickler & Ansprechpartner: inf19109@lehre.dhbw-stuttgart.de & gute und effiziente Umsetzung der Aufgabenstellung\\
\hline
\end{tabularx}
\caption{Stakegholder im Projekt eCourse}
\label{tab:Stakeholder}
\end{table}

