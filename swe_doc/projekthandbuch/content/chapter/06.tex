\chapter{Risikomanagement}

Zur Ermittlung potenzieller Risiken werden zu Projektbeginn mit dem ganzen Team die relevanten Risiken des Projekts identifiziert. Zur Einstufung der Risiken wurde im Team die Eintrittswahrscheinlichkeit des Risikos sowie die dazugehörige Auswirkung geschätzt. Dies führte zu den Ergebnissen in Tabelle \ref{tab:Risiken}. Im Zuge der Besprechung der Risiken wurde auch der Umgang mit diesen Risiken diskutiert. \\
Im Umgang mit einem Risiko können 4 Strategien verfolgt werden. Grob zusammengefasst können Risiken \textit{vermieden} werden, in dem das Projekt anders gestaltet wird, sie können \textit{transferiert} werden, indem das Risiko an einen Dritten delegiert wird. Des Weiteren können die Risiken \textit{abgeschwächt} werden, indem präventive Maßnahmen definiert werden und zuletzt können Risiken auch \textit{akzeptiert} werden. \\
In Tabelle \ref{tab:Risiken} werden in der letzten Spalte für das jeweilige Risiko Vermeidungs- bzw. Abschwächungsstrategien aufgeführt. Um die Risiken im Projekt klein zu halten wird über die gesamte Projektdauer versucht diese Maßnahmen umzusetzen. \\ 
Sind in dieser Spalte keine Maßnahmen aufgeführt, kann das Projektteam dem Risiko nicht direkt entgegenwirken und muss das Risiko akzeptieren. \\
Eine Delegation der Risiken ist für das Projektteam nicht möglich, da das Projektteam selbstständig für das Projekt zuständig ist.

\begin{table}
\centering
\tiny
\begin{tabularx}{\textwidth}{|l|l|X|l|l|X|} 
\hline
Ursache & Ausmaß & Risiko & Auswirkung & Chance & Vermeidung/Abschwächung \\
\hline
Anforderungen & Produkt & Ungenügende Anforderungsanalyse & ernst & hoch &  \\ 
\hline
 &  & Im Produkteinsatz stellt sich heraus  dass mit wachsender Benutzerzahl Einbrüche hinsichtlich der Performance einhergehen. Als Folge können nicht alle Funktionen Effektiv und effizient genutzt werden . Die Nutzvorteile sind eingeschränkt. & tolerierbar & hoch &  \\ 
\hline
Extern & Produkt & Höhere Gewalt beeinflusst das Projekt & tolerierbar & gering &  \\ 
\hline
Kommunikation & Team & Probleme werden nicht offen angesprochen & ernst & mittel & Offenheit im Team fördern \\ 
\hline
 &  & Probleme bei der Programmierung werden zu lange nicht angesprochen & ernst & mittel & Offenheit im Team fördern \\ 
\hline
 &  & Fehlende Dokumente & ernst & mittel & Dokumente als Tasks in Sprints aufnehmen \\ 
\hline
Management & Produkt & unrealistische Zeitplanung & ernst & hoch & Puffer einplanen \\ 
\hline
 &  & fehlende Dokumentation & ernst & hoch & Team zur Dokumentation anhalten \\ 
\hline
 &  & vergessene und nachträglich hinzugefügte Tasks & ernst & mittel & Daily Scrums \\ 
\hline
 &  & zu viel Kommunikation & gering & gering & Zeitmanagment in Meetings \\ 
\hline
 & Team & schlechter Informationsfluss & ernst & gering & Offenheit im Team fördern \\ 
\hline
 &  & Unklare Befugnisse: Es ist unklar wer die Autorität hat ein Projektziel durchzusetzen & tolerierbar & gering & Kommunikation, DailyScrums \\ 
\hline
Produktumfang & Produkt & Komplexität wird unterschätzt & ernst & mittel &  \\ 
\hline
 &  & Im Zeitverlauf über die Entwicklungsphasen wird keine klare Linie verfolgt sondern es gibt ständig neue und wechselnde Wünsche nach Funktionen; die an die SW gestellten Anforderungen werden kontinuierlich verändert. & niedrig & gering & Kommunikation Einhaltung des Lasten- und Pflichtenheftes \\ 
\hline
 &  & Nicht umsetzbares Design & ernst & mittel & Kommunikation  \\ 
\hline
Systemarchitektur & Produkt & Ungenügende Spezifikationen von Schnittstellen & ernst & hoch & Anforderungsanalyse \\ 
\hline
 &  & Komponenten arbeiten nicht zusammen & ernst & mittel &  \\ 
\hline
 &  & Nur teilweise erfahren SW-Entwickler & tolerierbar & hoch &  \\ 
\hline
 &  & Komponenten laufen nicht stabil & ernst & mittel &  \\ 
\hline
 &  & Probleme mit Projektmanagemensystem & ernst & gering & Alternativen bereithalten \\ 
\hline
 & Stakeholder & Sicherheitslücken & ernst & mittel & Saubere Implementierung \\ 
\hline
 &  & Testumgebung für Integrationstests sind nicht verfügbar & ernst & gering & Frühzeitig erkennen \\ 
\hline
Team & Produkt & Nur teilweise erfahren SW-Entwickler & ernst & hoch & Pairprogramming \\ 
\hline
 &  & Mangelhafte Zeitplanung & tolerierbar & mittel & Kommunikation  \\ 
\hline
 &  & Implementierung unnötiger Eigenschaften und keine Entwicklung einer der Anforderung entsprechenden Eigenschaft & mittel & gering & Offenheit im Team fördern \\ 
\hline
 &  & Nicht genügend Entwickler & ernst & mittel & Scope  \\ 
\hline
 &  & Austausch/ Wegfall von Personal (Exmatrikulation o. ä.) & mittel & gering &  \\ 
\hline
 &  & Einsatz neuer unbekannter Tools & tolerierbar & hoch & Wiki \\ 
\hline
 &  & Mitarbeiter müssen sich in Tools einarbeiten & tolerierbar & hoch & Kommunikation  \\ 
\hline
 &  & Mangelnde Leistung Engagement & ernst & mittel & Kommunikation \\ 
\hline
 &  & Negative Einstellung gegenüber des Projektes & ernst & gering & Offenheit im Team fördern \\ 
\hline
 & Team & Geringe Motivation & ernst & mittel & Zeitplan einhalten \\ 
\hline
\end{tabularx}
\caption{Risiken im Projekt eCourse}
\label{tab:Risiken}
\end{table}
