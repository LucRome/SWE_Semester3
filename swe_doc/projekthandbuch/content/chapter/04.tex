\chapter{Projektkommunikationsplan}

Zur internen Kommunikation innerhalb des Projektes wird die Plattform Discord verwendet. Dort wurde ein eigens für dieses Projekt bestimmter Server erstellt, auf welchem die Kommunikation stattfindet. \\
Alle Besprechungen, die das gesamte \gls{Development Team} betreffen werden im Kanal \glqq General\grqq\: abgehalten. Dazu zählt auch schriftliche Kommunikation via Chat. \\
Sind einzelne Inhalte nur für eine bestimmte Gruppe interessant, kann auf die anderen, entsprechend gekennzeichneten Kanäle ausgewichen werden. \\
In Tabelle \ref{tab:Kommunikation} sind alle regelmäßigen Projektmeetings aufgeführt. Zu diesen Meetings wird eine vollzählige Anwesenheit des \gls{Development Team}s erwartet.

\begin{table}[h]
\centering
\tiny
\begin{tabular} [h] {|c|c|c|c|}
\hline
Bezeichnung des Meetings & Inhalte & Häufigkeit und Dauer \\
\hline
\gls{Daily Scrum} & Besprechung was getan wurde & jeden Montag von 9.30 bis 10.00\\
\quad & und was noch getan wird & und jeden Mittwoch von 15.00 bis 15.30 \\
\hline
\gls{Sprint Review} & Besprechung der im Sprint geleisteten Arbeit & jeden Donnerstag direkt nach der SoftwareEngineering-Vorlesung\\
\hline
\gls{Sprint Retrospektive} & Besprechung des Ablaufs des letzten Sprints & jeden Donnerstag direkt nach dem \gls{Sprint Review}\\
\hline
\gls{Sprint Planning} & Planung des nächsten Sprints & jeden Donnerstag direkt nach der \gls{Sprint Retrospektive}\\
\hline
\end{tabular}
\caption{Projektkommunikationsplan}
\label{tab:Kommunikation}
\end{table}
