\chapter{Projektabschluss und Bewertung des Projekterfolges}

\label{sec:Chap7}

Im Nachfolgenden wird der Erfolg des Projektes sowie das Verbesserungspotential für kommende Projekte festgehalten.

\section{Erfolgsmessung}
Im Folgenden sind die im Lastenheft beschriebenen Anforderungen kurz zusammengefasst. Darauffolgend wird ein kurzes Fazit gezogen, welche Anforderungen umgesetzt werden konnten.

\textbf{Zielerreichung inhaltlich} \\
Gefordert:
\begin{itemize}
\item MUST:
\begin{itemize}
\item Drei Benutzergruppen
\item Login und Logout
\item Klausuren hochladen
\end{itemize}
\item SHOULD:
\begin{itemize}
\item Verwaltungsangestellte können Studenten zur Gruppen zusammenfassen
\item Namen der Kurse automatisch generieren
\item Dozierende können Bewertung für bearbeitete Abgaben der Studierenden eingeben
\end{itemize}
\item COULD:
\begin{itemize}
\item Dozierende werden nach gewisser Zeit an die Bewertung erinnert
\item Logo
\item Chatfunktion
\item Vorlesungsmaterialien hochladen
\end{itemize}
\end{itemize}

Es konnten alle Punkte der MUST-Kategorie umgesetzt werden. Aus Zeitgründen könnten die Ziele aus der SHOULD-Kategorie leider nicht umgesetzt werden. Allerdings konnten aus der COULD-Kategorie ein Logo umgesetzt werden und es ist auch möglich in eCourse Vorlesungsmaterialien hochzuladen.

\section{Reflektion/Lessons Learned}

\textbf{Team/Zusammenarbeit:}\\
\begin{itemize}
\item Gegenseitiger Respekt der geleisteten Arbeit ist wichtig 
\item Kommunikation zwischen den einzelnen Expertengruppen muss mehr gefördert werden
\end{itemize}

\textbf{Projektmanagement:}
\begin{itemize}
\item Kick-off Event durchführen
\item Zwischenziele festlegen
\end{itemize}

\textbf{Sonstiges:}\\
\begin{itemize}
\item Tests höher priorisieren (eventuell Test-Driven entwickeln)
\item Fragen direkt an betreffende Personen/Gruppe stellen, nicht warten bis zum DailyScrum
\end{itemize}

\section{Planung Nachprojektphase/Restaufgaben}

Da für dieses Projekt eine harte Deadline gesetzt wurde, können von diesem Projektteam keine weiteren Restaufgaben erledigt werden. \\
Allerdings ist es für ein Folgeprojekt möglich noch weitere Features in die Software zu implementieren. \\
Beispielsweise könnte die Bewertung der Abgaben noch implementiert werden oder eine Chatfunktion zwischen Studierenden und Dozierenden.