%!TEX root = ../../main.tex

\chapter{Anleitung für Verwaltungsangestellte}
\label{sec:chap1}
\textbf{Kursübersicht}
<<<<<<< Updated upstream

\textbf{Kurs anlegen}

\textbf{Kurs bearbeiten}
=======
Der Benutzer kann zu jedem Zeitpunkt zur Kursübersicht gelangen. Diese ist in der Menüleiste im Sandwichmenü zu den Kursen zu finden.\\
In der Kursübersicht findet der Benutzer alle Kurse. Diese sind je nach Anzahl auf verschiedene Seiten aufgeteilt. Ein Wechsel zwischen den Seiten kann durch klicken auf die jeweilige Seitenzahl erfolgen. 
\textcolor{magenta}{hier bild kursübersicht einfügen}

\textbf{Kurs anlegen}
Ebenfalls im Untermenü Kurse ist die Möglichkeit zum Kurse anlegen gegeben.\\
Ein Kurs kann angelegt werden, wenn das angezeigte Formular ausgefüllt wird. Es muss ein Dozent ausgewählt werden. Dieser kann im Dropdown-Menü aus einer Liste aller Dozenten ausgewählt werden. \\
Die Teilnehmer am Kurs aus der Gruppe der Studierenden werden hinzugefügt, indem man sie anhand ihres Benutzernamens in der Liste ausfindig macht und mit einem Haken versieht. \\
Außerdem sollte dem Kurs sinnvoller Name vergeben werden. Von den Entwicklern wird ein Name empfohlen der dem folgenden Schema entspricht: \\
\verb/Kursname_JJKursbezeichnung/
Abschließend kann über einen Kalender ein Start- und Enddatum des Kurses festgelegt werden. 
Die Erstellung des Kurses kann durch klicken der Schaltfläche \glqq Kurs erstellen\grqq\: beendet werden.

\textcolor{magenta}{bild mit der form}

\textbf{Kurs bearbeiten}
Kurse können nur bearbeitet werden, wenn sich der Benutzer in der Kursübersicht befindet. Dort findet sich neben jedem Kursnamen eine Schaltfläche \glqq bearbeiten\grqq . Wird diese betätigt öffnet sich das gleiche Formular wie auch bei der Kurserstellung. Im Gegensatz zum Anlegen des Kurses its das Formular aber hier mit den Informationen über den Kurs gefüllt und kann bei Bedarf abgeändert werden.\\
Wurden Änderungen vorgenommen, werden diese gespeichert in dem der Nutzer die Schaltfläche \glqq Speichern\grqq\: betätigt

